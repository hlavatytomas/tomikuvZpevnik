\sclearpage\beginsong{Cizinec}[by={Nedvědi}]
\beginverse
\[D]Za vesnici \[Hm]stával,
tam kde \[G]dobytek \[A]se pas 
nikdy s nikým \[F#m]nemluvil, 
nikdo \[Hm]neznal jeho hlas 
vždy \[G]zůstal jenom \[A7]na chvíli 
a \[D]důstojnou měl \[Hm]tvář 
často \[G]jen tak prošel \[A]polem, 
\[F#m7]jako dobrý hospodář\[Hm] 
a \[G]nikdo neví, \[A]odkud se tu \[D]vzal. 
\endverse
\beginchorus
\[A]Byla vám to záhadná \[D]věc
že \[G]kdykoliv se \[A]objevil 
ten \[D]divný \[Hm]cizinec 
\[A7]jako by nám \[A]tím chtěl něco \[D]říct 
\endchorus
 
\beginverse
Když viděli ho poprvé, 
držel v rukou meč a štít 
prý za blázna ho měli, 
ale nechali ho být 
a do týdne se vojna zvedla 
bůhví proč a zač 
jen rýchlo prošla krajem, 
v patách za ní křik a pláč 
a nikdo neví, kde se ten muž vzal. 
\endverse
 
\beginverse
Když jednou přišlo sucho 
nejhorší za sto let 
on objevil se znova, 
prošel ke studni a spět 
pak se v noci spustil liják, 
blesky bily tam a sem 
a přitom hvězdy zářily 
na nebi bezmračném 
a nikdo neví kde se ten déšť vzal. 
\endverse
 
\beginverse
Už dávno se tu neválčí 
a úrody je dost 
čas na práci i na lásku 
a důvod pro vděčnost 
to cizinec když před léty 
se zjevil naposled 
prý na tváři měl úsměv 
a v ruce bílý květ 
A pak už víckrát nevrátil se zpět. 
\endverse

\beginverse
Když mně to děda vyprávěl, 
tak zrovna kvetl věs 
a já si vzpomněl na dětství, 
vidím to jako dnes 
Jako kluk sem potkal lovce, 
nesl přes rameno síť 
pak i já svou první rybu chyt´. 
\endverse
 
\beginverse
A pak jsem jednou za bratrem 
až do Dublinu jel 
já na nedělní mši s ním 
tehdy do kostela šel 
Tam na zdi visel obraz, 
měl už oprýskaný rám 
ten muž co se z něj díval, 
to byl svatý Patrik sám 
a já najednou věděl, že ho znám. 
A já najednou věděl, že ho znám. 
\endverse
\endsong